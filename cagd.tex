\documentclass{article}
\usepackage[utf8x]{inputenc} % codifica scrittura
\usepackage[nochapters]{classicthesis} % nochapters

%\usepackage{inputenc}
\usepackage[T1]{fontenc} 
\usepackage[square,numbers]{natbib} 
\usepackage{amsmath, amsthm, amssymb, tikz}
\usetikzlibrary{snakes}
\newtheorem{definizione}{Definizione}  
\newtheorem{lemma}{Lemma}
\newtheorem{teorema}{Teorema}  
\newtheorem{proprieta}{Proprietà}  
\usepackage{titletoc}
\usepackage{verbatim}
\usepackage{hyperref}

\titlecontents{section}[3em]{}{\contentslabel{1em}}{}{\titlerule*[1.5pc]{.}\contentspage}
\titlecontents{subsection}[6em]{}{\contentslabel{2em}}{}{\titlerule*[1.5pc]{.}\contentspage}

\begin{document}

\title{\rmfamily\normalfont\spacedallcaps{Computer Aided Graphic
    Design course exercises}}

\author{\spacedlowsmallcaps{Massimo Nocentini}}
\date{\today} % no date

\maketitle


\begin{abstract}
  This document contains some exercises and collects work done during
  the CAGD course given by Prof. Alessandra Sestini and Prof. Costanza
  Conti at University of Florence.

  The content is organized in sections, each one of them correspond to
  a given set of exercises, the last one report the raw implementation
  code written in Julia \cite{Julia}.
\end{abstract}
       
\tableofcontents

\newpage

\section{Bezier curves}

\subsection{Curve from simple set of control points}
In \autoref{fig:first-closed-curve} we report the very first curve
obtained using our implementation reported in
\autoref{sec:deCasteljau-code}. The curve is obtained using control
points $(1,1), (3,4), (5,6),(7,8),(10,2),(1,1)$ in the given
order. This plot was the first test for our implementation of code
reported in Exercise 1 and requested in Exercise 2: it contains a
segmented curve in green which is the control polygon, and a Bezier
curve in red built using the given control points.
\begin{figure}
  \centering
  \includegraphics{bezier-deCasteljau-curves/exercise-one}  
  \caption{Curve from simple set of control points}
  \label{fig:first-closed-curve}
\end{figure}

\subsection{Curve from parametric specification}
As required in Exercise 3, in \autoref{fig:curve-from-parametric-spec}
we report a Bezier curve built from the parametric specification:
\begin{displaymath}
  \left [  \begin{array}{c}
      x(u) \\
      y(u)
    \end{array} \right ] = \left [  
    \begin{array}{c}
      1 + u + u^2 \\
      u^3
    \end{array} \right ]
\end{displaymath}
with $u\in[0,1]$. For plotting we choose to sample 6 control points,
uniformly spaced in $[0,1]$: observe that there isn't a relation
between the uniform sampling for the parameter $u$ and control points
distribution on the plane.

\begin{figure}
  \centering
  \includegraphics{bezier-deCasteljau-curves/exercise-two}
  \caption{Curve from parametric specification}
  \label{fig:curve-from-parametric-spec}
\end{figure}



\includegraphics{bezier-deCasteljau-curves/exercise-four}
\includegraphics{bezier-deCasteljau-curves/exercise-five}
\includegraphics{bezier-deCasteljau-curves/exercise-six-original}
\includegraphics{bezier-deCasteljau-curves/exercise-six-higher-degree-control-poly}
\includegraphics{bezier-deCasteljau-curves/exercise-six-one-more-degree-comparison}
\includegraphics{bezier-deCasteljau-curves/exercise-seven-continuity}
\includegraphics{bezier-deCasteljau-curves/exercise-seven-tangent}
\includegraphics{bezier-deCasteljau-curves/exercise-seven-obsculating}
\includegraphics{bezier-deCasteljau-curves/exercise-seven-a_succ_i}
\includegraphics{bezier-deCasteljau-curves/exercise-seven-continuity-left}
\includegraphics{bezier-deCasteljau-curves/exercise-seven-tangent-left}
\includegraphics{bezier-deCasteljau-curves/exercise-seven-obsculating-left}
\includegraphics{bezier-deCasteljau-curves/exercise-seven-a_i-left}

\section{Code}
\subsection{de Casteljau}
\label{sec:deCasteljau-code}
\verbatiminput{bezier-deCasteljau-curves/deCasteljau.jl}

\newpage

\begin{thebibliography}{}

\bibitem{Julia} Open Source Project,
  \emph{Julia language}, \url{http://julialang.org/}

\bibitem{ETH} Locher T., Y. A. Pignolet, R. Wattenhofer,
  \textit{Principles of Distribuited Computing}, Zurich, Swiss Federal
  Institute of Technology, 2013.


\end{thebibliography}



\end{document}
